\documentclass[10pt,a4paper]{article}

\usepackage[draft=false]{graphicx}
\usepackage{amsmath}
\usepackage{amssymb}
\usepackage{geometry}
\usepackage{multicol}
\usepackage{setspace}

\geometry{
left=1.2cm,
right=1.2cm,
top=1.2cm,
bottom=1.2cm
}

\setstretch{1.05}
\setlength{\columnsep}{0.8cm}
\pagestyle{empty}

\begin{document}

\begin{center}
\includegraphics[width=6cm]{logo.jpg}
\end{center}

\begin{center}
\textbf{NAME: SRINIDHI A}\\
\textbf{ID: COMETFWC053}
\end{center}

\begin{center}
\textbf{10th CBSE MATHEMATICS 2018}\\
Keshav Roy
\end{center}

\begin{multicols}{2}

\textbf{1 SECTION A}

\begin{enumerate}

\item Find the value of k for which the roots of a quadratic equation  
(k-5)x2 + 2(k-5)x + 2 = 0 are equal.

\item Find the value of y for which the distance between the points  
(2,-3) and (10,y) is 10 units.

\item Write whether the rational number 13/3125 has a decimal expansion
which is terminating or non-terminating repeating.

\item Write the nth term of the A.P  
1/k , (1+k)/k , (1+2k)/k , ...

\item If sinθ + cosθ = √2 cos(90° − θ),
find the value of cotθ.

\item DE is drawn parallel to the base BC of triangle ABC, meeting AB at D
and AC at E. If AB/CD = 4 and CE = 2 cm, find AE.

\end{enumerate}

\textbf{2 SECTION B}

\begin{enumerate}

\item A bag contains 5 red balls and some blue balls. If the probability of drawing
a blue ball from the bag is three times that of the red ball, find the number of
blue balls in the bag.

\item The 5th and 15th terms of an A.P are 13 and −17 respectively.
Find the sum of first 21 terms of the A.P.

\item Using Euclid’s Division Algorithm, find the HCF of 225 and 867.

\item If the point (0,2) is equidistant from the points (3,k) and (k,5),
find the value of k.

\item Find the value of a for which the pair of linear equations  
2x + 3y = 7 and 4x + ay = 14 has infinitely many solutions.

\end{enumerate}

\columnbreak

\textbf{3 SECTION C}

\begin{enumerate}

\item Show that any positive odd integer is of the form 4q + 1 or 4q + 3
for some integer q.

\item The ten’s digit of a number is twice its unit’s digit. The number obtained
by interchanging the digits is 36 less than the original number.
Find the original number.

\item (i) The line segment joining the points A(2,1) and B(5,−8) is trisected
at the points P and Q, where P is nearer to A. If P lies on the line
2x − y + k = 0, find the value of k.

\textbf{OR}

(ii) The x-coordinate of a point P is twice its y-coordinate. If P is
equidistant from the point Q(2,−5) and R(−3,6), find the coordinates.

\item Show that 1, 1/2 and −2 are the zeroes of the polynomial  
2x3 + x2 − 5x + 2.

\item Prove that the angle between the two tangents drawn from an external point
to a circle is supplementary to the angle subtended by the line segment joining
the points of contact at the centre.

\item S and T are points on the sides PR and QR of triangle PQR such that  
angle P = angle RTS. Show that triangle RPQ is similar to triangle RTS.

\end{enumerate}

\end{multicols}

\end{document}
